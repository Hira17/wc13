Skip to content
Search or jump to…
Pull requests
Issues
Codespaces
Marketplace
Explore
 
@Hira17 
CS-113-Spring-2023
/
wc13
Public template
Fork your own copy of CS-113-Spring-2023/wc13
Code
Issues
Pull requests
Actions
Projects
Security
Insights
You’re making changes in a project you don’t have write access to. We’ve created a fork of this project for you to commit your proposed changes to. Submitting a change will write it to a new branch in your fork, so you can send a pull request.
wc13
/
wc13.tex
in
CS-113-Spring-2023:main
 

Spaces

2

Soft wrap
1
\documentclass[a4paper]{exam}
2
​
3
\usepackage{amsmath}
4
\usepackage{amssymb}
5
\usepackage{array}
6
\usepackage{geometry}
7
\usepackage{hyperref}
8
\usepackage{titling}
9
\usepackage{graphicx}
10
\usepackage{float}
11
​
12
\graphicspath{{images/}}
13
​
14
\newcolumntype{C}{>{$}c<{$}} % math-mode version of "c" column type
15
​
16
\runningheader{CS/MATH 113}{WC13: Proof by Induction}{\theauthor}
17
\runningheadrule
18
\runningfootrule
19
\runningfooter{}{Page \thepage\ of \numpages}{}
20
​
21
\printanswers
22
​
23
\title{Weekly Challenge 13: Proof by Induction\\CS/MATH 113 Discrete Mathematics}
24
\author{team-name}  % <== for grading, replace with your team name, e.g. q1-team-420
25
\date{Habib University | Spring 2023}
26
​
27
\qformat{{\large\bf \thequestion. \thequestiontitle}\hfill}
28
\boxedpoints
29
​
30
\begin{document}
31
\maketitle
32
​
33
\begin{questions}
34
​
35
  \titledquestion{Pyramid Scheme}[5]
36
  Suppose we have circular coins (alot of them!) all of same dimension and we were to make a hexagonal pyramid out of them as follows. The top layer has $1$ coin. The second layer has $7$ coins arranged in a hexagon with side length of two coins (see picture below). The third layer has $19$ coins in a hexagon with side length of three coins, and so on. Prove that a pyramid created so constituting $n$ layers requires in total $n^3$ coins.
37
  \begin{figure}[h!]
38
    \centerline{\includegraphics{layer1.png}}
39
    \caption{Top layer as seen from above.}
40
    \label{layer1}
41
  \end{figure}
42
  \begin{figure}[h!]
43
    \centerline{\includegraphics{layer2.png}}
44
    \caption{The second layer as seen from above. Note that this layer has $7$ coins and forms a hexagon with side length of two coins.}
45
    \label{layer2}
@Hira17
Propose changes
Commit summary
Create wc13.tex
Optional extended description
Add an optional extended description…
 
Footer
© 2023 GitHub, Inc.
Footer navigation
Terms
Privacy
Security
Status
Docs
Contact GitHub
Pricing
API
Training
Blog
About
