\documentclass[a4paper]{exam}

\usepackage{amsmath}
\usepackage{amssymb}
\usepackage{array}
\usepackage{geometry}
\usepackage{hyperref}
\usepackage{titling}
\usepackage{graphicx}
\usepackage{float}

\graphicspath{{images/}}

\newcolumntype{C}{>{$}c<{$}} % math-mode version of "c" column type

\runningheader{CS/MATH 113}{WC13: Proof by Induction}{\theauthor}
\runningheadrule
\runningfootrule
\runningfooter{}{Page \thepage\ of \numpages}{}

\printanswers

\title{Weekly Challenge 13: Proof by Induction\\CS/MATH 113 Discrete Mathematics}
\author{team-name}  % <== for grading, replace with your team name, e.g. q1-team-420
\date{Habib University | Spring 2023}

\qformat{{\large\bf \thequestion. \thequestiontitle}\hfill}
\boxedpoints

\begin{document}
\maketitle

\begin{questions}

  \titledquestion{Pyramid Scheme}[5]
  Suppose we have circular coins (alot of them!) all of same dimension and we were to make a hexagonal pyramid out of them as follows. The top layer has $1$ coin. The second layer has $7$ coins arranged in a hexagon with side length of two coins (see picture below). The third layer has $19$ coins in a hexagon with side length of three coins, and so on. Prove that a pyramid created so constituting $n$ layers requires in total $n^3$ coins.
  \begin{figure}[h!]
    \centerline{\includegraphics{layer1.png}}
    \caption{Top layer as seen from above.}
    \label{layer1}
  \end{figure}
  \begin{figure}[h!]
    \centerline{\includegraphics{layer2.png}}
    \caption{The second layer as seen from above. Note that this layer has $7$ coins and forms a hexagon with side length of two coins.}
    \label{layer2}
  \end{figure}
  \newpage
  \begin{figure}[h!]
    \centerline{\includegraphics{layer3.png}}
    \caption{The third layer with $19$ coins forming a hexagon with side length of three coins.}
    \label{layer3}
  \end{figure}
  
  \begin{solution}
    we can prove this through induction, for this we will first consider the base case that is when we have n=1 layer we will have one coin as according to the formula $n^3$ the cube of 1 is 1. Thus this tells that the base case is True.
    In the inductive hypotheses, we will assume that the total 
    number of coins in k layers will be equal to $k^3$. 
    \\
    So according to the pattern, it can be viewed that according to this formula $3n^2+3n+1$ we can know the number of coins in the next layer. 
    \\
    in the inductive step, we will prove that in k+1 layers we will have $(k+1)^3$ of coins. $k^3$ will give us the number of coins in the previous layers and then we will find the ${k+1}^{th}$ layers of coins through the formula $3n^2+3n+1$ and we will add them together. The answer will be $k^3+3k^2+3k+1$. When we will expand $(k+1)^3$ we will get the same term as $k^3+3k^2+3k+1$. Thus our inductive step is proven that is $n^3$ is equal to $n^3$.
  
  \end{solution}

  Note: All figures are taken from beckykwarren's geogebra \href{https://www.geogebra.org/m/cnqdjcph}{page.}

\end{questions}


\end{document}



%%% Local Variables:
%%% mode: latex
%%% TeX-master: t
%%% End:
